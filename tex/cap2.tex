%%%%%%%%%%%%%%%%%%%%%%%%%%%%%%%%%%%%%%%%%%%%%%%%%%%%%%%%%%%%%%%%%%%%%%%%%%%%%%%
% Chapter 2: The simulation optimization model requisitos : 
%%%%%%%%%%%%%%%%%%%%%%%%%%%%%%%%%%%%%%%%%%%%%%%%%%%%%%%%%%%%%%%%%%%%%%%%%%%%%%%

%++++++++++++++++++++++++++++++++++++++++++++++++++++++++++++++++++++++++++++++
In the berth planning problem, we should answer the following questions:
 how many segments do we organize for active shipping services? 
 how many cranes – out of the total, fixed, number of available ones  do we allocate for each of the organized segments? 
 to which segment do we forward incoming vessels, provided that we may base this decision on some suitable attributes shared by any given subset of the active services? 
Answers to these questions are provided by the simulation optimization approach.
3.2. Grid enabled versions for the SARP algorithm
With the aim of developing grid versions of the SARP algorithm described above (in the sequel G-SARP),
we have adopted a master/worker approach [16,4,19].
Let p be the number of workers. At iteration k, the master creates p + 1 perturbations of a same configuration
of the system (i.e. p + 1 neighbours of the current solution), keeps for itself the first generated configuration
to be estimated and sends the others to the workers (one configuration for each worker). Each processor,
including the master, must decide if its own configuration can be accepted or not. If a processor accepts its
own solution (according to the acceptance criterion of SARP), then sends it to the master. If the master has
new solutions to examine, it selects the next configuration, otherwise it keeps the old one. There are different
rules for selection; for example, the master can choose the configuration with the best estimated performance
(best strategy) or it can make a random choice (random strategy). We have implemented both, and we present in
 4 and 5 illustrate the master and the worker process, respectively. In
our computational scheme, different workers carry out simulation experiments on different neighbours.
In other words, through the grid platform, we explore a richer neighbourhood at each iteration and perform
a more effective procedure for optimization via simulation; whereas SARP simply uses a simulated
annealing routine, G-SARP integrates simulated annealing with a sampling routine. To this purpose, a point
of synchronization between the master and the workers is needed to evaluate the results of sampling, update
current solution and redefine the neighbours.
Our procedure has been tuned as follows.
%++++++++++++++++++++++++++++++++++++++++++++++++++++++++++++++++++++++++++++++

%\section{Primer apartado del segundo capítulo}
%\label{2:sec:1}
%  Primer párrafo de la primera sección.

%\section{Segundo apartado del segundo capítulo}
%\label{2:sec:2}
%  Primer párrafo de la segunda sección.

