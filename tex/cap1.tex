%%%%%%%%%%%%%%%%%%%%%%%%%%%%%%%%%%%%%%%%%%%%%%%%%%%%%%%%%%%%%%%%%%%%%%%%%%%%%
% Chapter 1: Queuing model and mathematical formulation
%%%%%%%%%%%%%%%%%%%%%%%%%%%%%%%%%%%%%%%%%%%%%%%%%%%%%%%%%%%%%%%%%%%%%%%%%%%%%%%
In the sequel we illustrate the proposed queuing model in which we eliminate some details from the model description of Legato and Mazza [12] because they are not significant for the purpose of this paper.
We assume that the occurrence of a delay-time spent at roadstead by an incoming vessel, due to lack of berth slots, is represented as a special case of the phase type Cox’s distribution. This assumption can be accepted for the following considerations. Usually, an incoming vessel receives almost immediately the required slots for berthing. In this case, with a very high probability (p), the elapsing interval from ‘‘arrival to port’’ and ‘‘berthing time completion’’ results in a very short time (l1, on average), due to the relatively fast operations for vessel positioning along the berth. With a very low probability (1  p), an arrived vessel is delayed at roadstead due to unavailability of the required berth slots: once this happens, then a very long interval (l1 + l2, on average) elapses from ‘‘arrival to port’’ and ‘‘berthing time’’ (Fig. 1). An illustrative example based on real data referred to a shipping company at Gioia Tauro port, in September 2002, is given in Fig. 2. Fig.
In the sequel we illustrate the proposed queuing model in which we eliminate some details from the model
description of Legato and Mazza [12] because they are not significant for the purpose of this paper.
We assume that the occurrence of a delay-time spent at roadstead by an incoming vessel, due to lack of
berth slots, is represented as a special case of the phase type Cox’s distribution. This assumption can be
accepted for the following considerations. Usually, an incoming vessel receives almost immediately the
required slots for berthing. In this case, with a very high probability (p), the elapsing interval from ‘‘arrival
to port’’ and ‘‘berthing time completion’’ results in a very short time (l1, on average), due to the relatively
fast operations for vessel positioning along the berth. With a very low probability (1  p), an arrived vessel
is delayed at roadstead due to unavailability of the required berth slots: once this happens, then a very long
interval (l1 + l2, on average) elapses from ‘‘arrival to port’’ and ‘‘berthing time’’ (Fig. 1). An illustrative
example based on real data referred to a shipping company at Gioia Tauro port, in September 2002, is given
in Fig. 2.


%---------------------------------------------------------------------------------
%\section{Sección Uno}
%\label{1:sec:1}
 % Primer párrafo de la primera sección.


%---------------------------------------------------------------------------------
%\section{Sección Dos}
%\label{1:sec:2}
%  Primer párrafo de la segunda sección.

%\begin{itemize}
%  \item Item 1
%  \item Item 2
%  \item Item 3
%\end{itemize}

