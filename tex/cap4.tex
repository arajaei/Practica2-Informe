%%%%%%%%%%%%%%%%%%%%%%%%%%%%%%%%%%%%%%%%%%%%%%%%%%%%%%%%%%%%%%%%%%%%%%%%%%%%%
% Chapter 4: Conclusion and future work
%%%%%%%%%%%%%%%%%%%%%%%%%%%%%%%%%%%%%%%%%%%%%%%%%%%%%%%%%%%%%%%%%%%%%%%%%%%%%%%

We have designed and implemented on a grid platform a simulation optimization algorithm that requires a limited amount of communication and a minimal synchronization among computing resources. It could be used for a cost-effective solution of logistics decision problems at seaport container terminals. Numerical evidence shows that as the number of workers increases, the quality of the final solution improves. A crucial role is played by the optimal compromise between searching the entire feasible region (exploration) and locally searching promising sub-regions (exploitation).
Numerical results are presented to compare the quality of solutions for different number of processors
employed, taking as reference 14 test problems which differ in the following parameters: profit values, number
of shipping services, maximum number of berth segments and budget. In particular, simulation runs have been
executed without resorting to any parallel computing capability, even though this could be implemented
according to the guidelines in Section 4. Here we concentrate on isolating the trend of improvement of the
neighbourhood exploration algorithm (simulated annealing with sampling) as the number of workers
increases, under both the alternative strategies (best and random) for updating the current solution. The
improvement of the quality of the final solution appears more significant as the complexity of the test problem
increases. In this case, the practical possibility of exploring a richer sub-region of the solution space increases
substantially by using a larger number of workers. Hence, an interesting research issue raised by the master–
worker framework is that of finding the most effective balance between the explorative work carried out by
the workers (all together) on an intelligent partition of the entire feasible region and the exploitative work
by one worker within one sub-region.
